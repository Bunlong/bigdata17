\documentclass[]{book}
\usepackage{lmodern}
\usepackage{amssymb,amsmath}
\usepackage{ifxetex,ifluatex}
\usepackage{fixltx2e} % provides \textsubscript
\ifnum 0\ifxetex 1\fi\ifluatex 1\fi=0 % if pdftex
  \usepackage[T1]{fontenc}
  \usepackage[utf8]{inputenc}
\else % if luatex or xelatex
  \ifxetex
    \usepackage{mathspec}
  \else
    \usepackage{fontspec}
  \fi
  \defaultfontfeatures{Ligatures=TeX,Scale=MatchLowercase}
\fi
% use upquote if available, for straight quotes in verbatim environments
\IfFileExists{upquote.sty}{\usepackage{upquote}}{}
% use microtype if available
\IfFileExists{microtype.sty}{%
\usepackage{microtype}
\UseMicrotypeSet[protrusion]{basicmath} % disable protrusion for tt fonts
}{}
\usepackage[margin=1in]{geometry}
\usepackage{hyperref}
\hypersetup{unicode=true,
            pdftitle={Computing for Big Data (BST-262)},
            pdfauthor={Christine Choirat},
            pdfborder={0 0 0},
            breaklinks=true}
\urlstyle{same}  % don't use monospace font for urls
\usepackage{natbib}
\bibliographystyle{apalike}
\usepackage{color}
\usepackage{fancyvrb}
\newcommand{\VerbBar}{|}
\newcommand{\VERB}{\Verb[commandchars=\\\{\}]}
\DefineVerbatimEnvironment{Highlighting}{Verbatim}{commandchars=\\\{\}}
% Add ',fontsize=\small' for more characters per line
\usepackage{framed}
\definecolor{shadecolor}{RGB}{248,248,248}
\newenvironment{Shaded}{\begin{snugshade}}{\end{snugshade}}
\newcommand{\KeywordTok}[1]{\textcolor[rgb]{0.13,0.29,0.53}{\textbf{{#1}}}}
\newcommand{\DataTypeTok}[1]{\textcolor[rgb]{0.13,0.29,0.53}{{#1}}}
\newcommand{\DecValTok}[1]{\textcolor[rgb]{0.00,0.00,0.81}{{#1}}}
\newcommand{\BaseNTok}[1]{\textcolor[rgb]{0.00,0.00,0.81}{{#1}}}
\newcommand{\FloatTok}[1]{\textcolor[rgb]{0.00,0.00,0.81}{{#1}}}
\newcommand{\ConstantTok}[1]{\textcolor[rgb]{0.00,0.00,0.00}{{#1}}}
\newcommand{\CharTok}[1]{\textcolor[rgb]{0.31,0.60,0.02}{{#1}}}
\newcommand{\SpecialCharTok}[1]{\textcolor[rgb]{0.00,0.00,0.00}{{#1}}}
\newcommand{\StringTok}[1]{\textcolor[rgb]{0.31,0.60,0.02}{{#1}}}
\newcommand{\VerbatimStringTok}[1]{\textcolor[rgb]{0.31,0.60,0.02}{{#1}}}
\newcommand{\SpecialStringTok}[1]{\textcolor[rgb]{0.31,0.60,0.02}{{#1}}}
\newcommand{\ImportTok}[1]{{#1}}
\newcommand{\CommentTok}[1]{\textcolor[rgb]{0.56,0.35,0.01}{\textit{{#1}}}}
\newcommand{\DocumentationTok}[1]{\textcolor[rgb]{0.56,0.35,0.01}{\textbf{\textit{{#1}}}}}
\newcommand{\AnnotationTok}[1]{\textcolor[rgb]{0.56,0.35,0.01}{\textbf{\textit{{#1}}}}}
\newcommand{\CommentVarTok}[1]{\textcolor[rgb]{0.56,0.35,0.01}{\textbf{\textit{{#1}}}}}
\newcommand{\OtherTok}[1]{\textcolor[rgb]{0.56,0.35,0.01}{{#1}}}
\newcommand{\FunctionTok}[1]{\textcolor[rgb]{0.00,0.00,0.00}{{#1}}}
\newcommand{\VariableTok}[1]{\textcolor[rgb]{0.00,0.00,0.00}{{#1}}}
\newcommand{\ControlFlowTok}[1]{\textcolor[rgb]{0.13,0.29,0.53}{\textbf{{#1}}}}
\newcommand{\OperatorTok}[1]{\textcolor[rgb]{0.81,0.36,0.00}{\textbf{{#1}}}}
\newcommand{\BuiltInTok}[1]{{#1}}
\newcommand{\ExtensionTok}[1]{{#1}}
\newcommand{\PreprocessorTok}[1]{\textcolor[rgb]{0.56,0.35,0.01}{\textit{{#1}}}}
\newcommand{\AttributeTok}[1]{\textcolor[rgb]{0.77,0.63,0.00}{{#1}}}
\newcommand{\RegionMarkerTok}[1]{{#1}}
\newcommand{\InformationTok}[1]{\textcolor[rgb]{0.56,0.35,0.01}{\textbf{\textit{{#1}}}}}
\newcommand{\WarningTok}[1]{\textcolor[rgb]{0.56,0.35,0.01}{\textbf{\textit{{#1}}}}}
\newcommand{\AlertTok}[1]{\textcolor[rgb]{0.94,0.16,0.16}{{#1}}}
\newcommand{\ErrorTok}[1]{\textcolor[rgb]{0.64,0.00,0.00}{\textbf{{#1}}}}
\newcommand{\NormalTok}[1]{{#1}}
\usepackage{longtable,booktabs}
\usepackage{graphicx,grffile}
\makeatletter
\def\maxwidth{\ifdim\Gin@nat@width>\linewidth\linewidth\else\Gin@nat@width\fi}
\def\maxheight{\ifdim\Gin@nat@height>\textheight\textheight\else\Gin@nat@height\fi}
\makeatother
% Scale images if necessary, so that they will not overflow the page
% margins by default, and it is still possible to overwrite the defaults
% using explicit options in \includegraphics[width, height, ...]{}
\setkeys{Gin}{width=\maxwidth,height=\maxheight,keepaspectratio}
\IfFileExists{parskip.sty}{%
\usepackage{parskip}
}{% else
\setlength{\parindent}{0pt}
\setlength{\parskip}{6pt plus 2pt minus 1pt}
}
\setlength{\emergencystretch}{3em}  % prevent overfull lines
\providecommand{\tightlist}{%
  \setlength{\itemsep}{0pt}\setlength{\parskip}{0pt}}
\setcounter{secnumdepth}{5}
% Redefines (sub)paragraphs to behave more like sections
\ifx\paragraph\undefined\else
\let\oldparagraph\paragraph
\renewcommand{\paragraph}[1]{\oldparagraph{#1}\mbox{}}
\fi
\ifx\subparagraph\undefined\else
\let\oldsubparagraph\subparagraph
\renewcommand{\subparagraph}[1]{\oldsubparagraph{#1}\mbox{}}
\fi

%%% Use protect on footnotes to avoid problems with footnotes in titles
\let\rmarkdownfootnote\footnote%
\def\footnote{\protect\rmarkdownfootnote}

%%% Change title format to be more compact
\usepackage{titling}

% Create subtitle command for use in maketitle
\newcommand{\subtitle}[1]{
  \posttitle{
    \begin{center}\large#1\end{center}
    }
}

\setlength{\droptitle}{-2em}
  \title{Computing for Big Data (BST-262)}
  \pretitle{\vspace{\droptitle}\centering\huge}
  \posttitle{\par}
  \author{Christine Choirat}
  \preauthor{\centering\large\emph}
  \postauthor{\par}
  \predate{\centering\large\emph}
  \postdate{\par}
  \date{2017-09-15}

\usepackage{booktabs}
\usepackage{amsthm}
\makeatletter
\def\thm@space@setup{%
  \thm@preskip=8pt plus 2pt minus 4pt
  \thm@postskip=\thm@preskip
}
\makeatother

\usepackage{amsthm}
\newtheorem{theorem}{Theorem}[chapter]
\newtheorem{lemma}{Lemma}[chapter]
\theoremstyle{definition}
\newtheorem{definition}{Definition}[chapter]
\newtheorem{corollary}{Corollary}[chapter]
\newtheorem{proposition}{Proposition}[chapter]
\theoremstyle{definition}
\newtheorem{example}{Example}[chapter]
\theoremstyle{definition}
\newtheorem{exercise}{Exercise}[chapter]
\theoremstyle{remark}
\newtheorem*{remark}{Remark}
\newtheorem*{solution}{Solution}
\begin{document}
\maketitle

{
\setcounter{tocdepth}{1}
\tableofcontents
}
\chapter{Introduction}\label{intro}

\section{Prerequisites}\label{prerequisites}

For BST262 (Computing for Big Data), we assume familiarity with the
material covered in BST260 (Introductio to Data Science).

We will use R to present concepts that are mostly language-agnostic. We
could have used Python, as in BST261 (Data Science II).

\section{Syllabus}\label{syllabus}

Week 1 - Basic tools

\begin{itemize}
\tightlist
\item
  Lecture 1. Unix scripting, make
\item
  Lecture 2. Version control: Git and GitHub (guest lecture: Ista Zhan)
\end{itemize}

Week 2 - Creating and maintaining R packages

\begin{itemize}
\tightlist
\item
  Lecture 3. Rationale, package structure, available tools
\item
  Lecture 4. Basics of software engineering: unit testing, continuous
  integration, code coverage
\end{itemize}

Week 3 - Software optimization

\begin{itemize}
\tightlist
\item
  Lecture 5. Measuring performance: profiling and benchmarking tools
\item
  Lecture 6. Improving performance: an introduction to C/C++, Rcpp
\end{itemize}

Week 4 -- Databases

\begin{itemize}
\tightlist
\item
  Lecture 7. Overview of SQL (SQLite, PostgreSQL) and noSQL databases
  (HBase, MongoDB, Cassandra, BigTable, \ldots{})
\item
  Lecture 8. R database interfaces (in particular through dplyr)
\end{itemize}

Week 5 - Analyzing data that does not fit in memory

\begin{itemize}
\tightlist
\item
  Lecture 9. Pure R solutions (sampling, ff and ffbase, other
  interpreters). JVM solutions (h20, Spark)
\item
  Lecture 10. An introduction to parallel computing; clusters and cloud
  computing. ``Divide and Conquer'' (MapReduce approaches)
\end{itemize}

Week 6 -- Visualization

\begin{itemize}
\tightlist
\item
  Lecture 11. Principles of visualization (guest lecture: James Honaker)
\item
  Lecture 12. Maps and GIS: principles of GIS, using R as a GIS, PostGIS
\end{itemize}

Weeks 7 \& 8 - Guest lectures (order and precise schedule TBD)

\begin{itemize}
\tightlist
\item
  Software project management (Danny Brooke)
\item
  R and Spark (Ellen Kraffmiller and Robert Treacy) Advanced GIS and
  remote sensing (TBD)
\item
  Cluster architecture (William J. Horka)
\end{itemize}

\section{Evaluation}\label{evaluation}

Grades will be based on two mandatory problem sets. Each problem set
will correspond to 50\% (= 50 points) of the final grade. The first
problem set will be available by the end of week 3 and the second
problem set by the end of week 6.

You will be required to submit problem set solutions within two weeks.
Grades, and feedback when appropriate, will be returned two weeks after
submission.

You will submit a markdown document that combines commented code for
data analysis and detailed and structured explanations of the algorithms
and software tools that you used.

\section{Software packages}\label{software-packages}

We will mostly use R in this course. Some examples will be run in
Python. In genral, we will use free and open-source software programs
such as PostgreSQL or Spark.

\section{Datasets}\label{datasets}

\subsection{MovieLens}\label{movielens}

MovieLens by
\citet[\url{https://grouplens.org/datasets/movielens/}]{Harper2015}
collects datasets from the website \url{https://movielens.org/}.

There are datasets of different sizes, notably:

\begin{enumerate}
\def\labelenumi{\arabic{enumi}.}
\tightlist
\item
  Small (1MB)
\item
  Benchmark (\textasciitilde{}190MB zipped)
\end{enumerate}

\begin{Shaded}
\begin{Highlighting}[]
\NormalTok{ratings <-}\StringTok{ }\KeywordTok{read.csv}\NormalTok{(}\StringTok{"data/ml-latest-small/ratings.csv"}\NormalTok{)}
\KeywordTok{head}\NormalTok{(ratings)}
\end{Highlighting}
\end{Shaded}

\begin{verbatim}
##   userId movieId rating  timestamp
## 1      1      31    2.5 1260759144
## 2      1    1029    3.0 1260759179
## 3      1    1061    3.0 1260759182
## 4      1    1129    2.0 1260759185
## 5      1    1172    4.0 1260759205
## 6      1    1263    2.0 1260759151
\end{verbatim}

\begin{Shaded}
\begin{Highlighting}[]
\NormalTok{links <-}\StringTok{ }\KeywordTok{read.csv}\NormalTok{(}\StringTok{"data/ml-latest-small/links.csv"}\NormalTok{)}
\KeywordTok{head}\NormalTok{(links)}
\end{Highlighting}
\end{Shaded}

\begin{verbatim}
##   movieId imdbId tmdbId
## 1       1 114709    862
## 2       2 113497   8844
## 3       3 113228  15602
## 4       4 114885  31357
## 5       5 113041  11862
## 6       6 113277    949
\end{verbatim}

\begin{Shaded}
\begin{Highlighting}[]
\NormalTok{movies <-}\StringTok{ }\KeywordTok{read.csv}\NormalTok{(}\StringTok{"data/ml-latest-small/movies.csv"}\NormalTok{)}
\KeywordTok{head}\NormalTok{(movies)}
\end{Highlighting}
\end{Shaded}

\begin{verbatim}
##   movieId                              title
## 1       1                   Toy Story (1995)
## 2       2                     Jumanji (1995)
## 3       3            Grumpier Old Men (1995)
## 4       4           Waiting to Exhale (1995)
## 5       5 Father of the Bride Part II (1995)
## 6       6                        Heat (1995)
##                                        genres
## 1 Adventure|Animation|Children|Comedy|Fantasy
## 2                  Adventure|Children|Fantasy
## 3                              Comedy|Romance
## 4                        Comedy|Drama|Romance
## 5                                      Comedy
## 6                       Action|Crime|Thriller
\end{verbatim}

\begin{Shaded}
\begin{Highlighting}[]
\NormalTok{tags <-}\StringTok{ }\KeywordTok{read.csv}\NormalTok{(}\StringTok{"data/ml-latest-small/tags.csv"}\NormalTok{)}
\KeywordTok{head}\NormalTok{(tags)}
\end{Highlighting}
\end{Shaded}

\begin{verbatim}
##   userId movieId                     tag  timestamp
## 1     15     339 sandra 'boring' bullock 1138537770
## 2     15    1955                 dentist 1193435061
## 3     15    7478                Cambodia 1170560997
## 4     15   32892                 Russian 1170626366
## 5     15   34162             forgettable 1141391765
## 6     15   35957                   short 1141391873
\end{verbatim}

\subsection{Airlines data}\label{airlines-data}

\subsection{Census}\label{census}

\subsection{Health claims}\label{health-claims}

\subsection{\texorpdfstring{GIS: PM\textsubscript{2.5}
exposure}{GIS: PM2.5 exposure}}\label{gis-pm2.5-exposure}

PM\textsubscript{2.5} exposure

\subsection{Methylation?}\label{methylation}

\subsection{Genomics??}\label{genomics}

\subsection{GWAS}\label{gwas}

\chapter{Basic tools}\label{basics}

\section{Command line tools}\label{command-line-tools}

\section{Git and GitHub}\label{git-and-github}

\chapter{Packages}\label{packages}

\section{Why?}\label{why}

\begin{itemize}
\item
  Organize your code
\item
  Distribute your code
\item
  Keep versions of your code
\end{itemize}

\section{Structure}\label{structure}

\begin{itemize}
\tightlist
\item
  Folder hierarchy

  \begin{itemize}
  \tightlist
  \item
    \texttt{NAMESPACE}: package import / export
  \item
    \texttt{DESCRIPTION}: metadata
  \item
    \texttt{R/}: R code
  \item
    \texttt{man/}: object documentation (with short examples)
  \item
    \texttt{tests/}
  \item
    \texttt{data/}
  \item
    \texttt{src/}: compiled code
  \item
    \texttt{vignettes/}: manual-like documentation
  \item
    \texttt{inst/}: installed files
  \item
    \texttt{demo/}: longer examples
  \item
    \texttt{exec}, \texttt{po}, \texttt{tools}
  \end{itemize}
\end{itemize}

\section{Building steps}\label{building-steps}

\begin{itemize}
\item
  \texttt{R\ CMD\ build}
\item
  \texttt{R\ CMD\ INSTALL}
\item
  \texttt{R\ CMD\ check}
\end{itemize}

\section{\texorpdfstring{\texttt{R\ CMD\ build}}{R CMD build}}\label{r-cmd-build}

\begin{Shaded}
\begin{Highlighting}[]
\NormalTok{R CMD build --help}
\end{Highlighting}
\end{Shaded}

\emph{Build R packages from package sources in the directories specified
by `pkgdirs'}

\section{\texorpdfstring{\texttt{R\ CMD\ INSTALL}}{R CMD INSTALL}}\label{r-cmd-install}

\begin{Shaded}
\begin{Highlighting}[]
\NormalTok{R CMD INSTALL --help}
\end{Highlighting}
\end{Shaded}

\emph{Install the add-on packages specified by pkgs. The elements of
pkgs can be relative or absolute paths to directories with the package
sources, or to gzipped package `tar' archives. The library tree to
install to can be specified via `--library'. By default, packages are
installed in the library tree rooted at the first directory in
.libPaths() for an R session run in the current environment.}

\section{\texorpdfstring{\texttt{R\ CMD\ check}}{R CMD check}}\label{r-cmd-check}

\begin{Shaded}
\begin{Highlighting}[]
\NormalTok{R CMD check --help}
\end{Highlighting}
\end{Shaded}

\url{http://r-pkgs.had.co.nz/check.html}

\emph{Check R packages from package sources, which can be directories or
package `tar' archives with extension `.tar.gz', `.tar.bz2', `.tar.xz'
or `.tgz'.}

\emph{A variety of diagnostic checks on directory structure, index and
control files are performed. The package is installed into the log
directory and production of the package PDF manual is tested. All
examples and tests provided by the package are tested to see if they run
successfully. By default code in the vignettes is tested, as is
re-building the vignette PDFs.}

\section{\texorpdfstring{Building steps with
\texttt{devtools}}{Building steps with devtools}}\label{building-steps-with-devtools}

\begin{itemize}
\item
  \texttt{devtools::build}
\item
  \texttt{devtools::install}
\item
  \texttt{devtools::check}
\item
  and many others: \texttt{load\_all}, \texttt{document}, \texttt{test},
  \texttt{run\_examples}, \ldots{}
\end{itemize}

\section{Creating an R package}\label{creating-an-r-package}

\subsection{\texorpdfstring{\texttt{utils::package.skeleton}}{utils::package.skeleton}}\label{utilspackage.skeleton}

\begin{Shaded}
\begin{Highlighting}[]
\KeywordTok{package.skeleton}\NormalTok{() }\CommentTok{# "in "fresh" session ("anRpackage")}
\KeywordTok{package.skeleton}\NormalTok{(}\StringTok{"pkgname"}\NormalTok{) }\CommentTok{# in "fresh" session}

\KeywordTok{set.seed}\NormalTok{(}\DecValTok{02138}\NormalTok{)}
\NormalTok{f <-}\StringTok{ }\NormalTok{function(x, y) x+y}
\NormalTok{g <-}\StringTok{ }\NormalTok{function(x, y) x-y}
\NormalTok{d <-}\StringTok{ }\KeywordTok{data.frame}\NormalTok{(}\DataTypeTok{a =} \DecValTok{1}\NormalTok{, }\DataTypeTok{b =} \DecValTok{2}\NormalTok{)}
\NormalTok{e <-}\StringTok{ }\KeywordTok{rnorm}\NormalTok{(}\DecValTok{1000}\NormalTok{)}
\KeywordTok{package.skeleton}\NormalTok{(}\DataTypeTok{list =} \KeywordTok{c}\NormalTok{(}\StringTok{"f"}\NormalTok{,}\StringTok{"g"}\NormalTok{,}\StringTok{"d"}\NormalTok{,}\StringTok{"e"}\NormalTok{), }\DataTypeTok{name =} \StringTok{"pkgname"}\NormalTok{)}
\end{Highlighting}
\end{Shaded}

\subsection{\texorpdfstring{\texttt{devtools::create}}{devtools::create}}\label{devtoolscreate}

\begin{Shaded}
\begin{Highlighting}[]
\NormalTok{devtools::}\KeywordTok{create}\NormalTok{(}\StringTok{"path/to/package/pkgname"}\NormalTok{)}
\end{Highlighting}
\end{Shaded}

\section{Submitting to CRAN}\label{submitting-to-cran}

\url{http://r-pkgs.had.co.nz/release.html}

\section{Using GitHub}\label{using-github}

\url{http://r-pkgs.had.co.nz/git.html}

\section{RStudio and GitHub integration (1 /
7)}\label{rstudio-and-github-integration-1-7}

\section{RStudio and GitHub integration (2 /
7)}\label{rstudio-and-github-integration-2-7}

\section{RStudio and GitHub integration (3 /
7)}\label{rstudio-and-github-integration-3-7}

\section{RStudio and GitHub integration (4 /
7)}\label{rstudio-and-github-integration-4-7}

\section{RStudio and GitHub integration (5 /
7)}\label{rstudio-and-github-integration-5-7}

\section{Command line}\label{command-line}

\begin{Shaded}
\begin{Highlighting}[]
\NormalTok{git init}
\NormalTok{git add *}
\NormalTok{git commit -m }\StringTok{"First commit"}
\NormalTok{git remote add origin git@github.com:harvard-P01/pkgtemplate.git}
\NormalTok{git push -u origin master}
\end{Highlighting}
\end{Shaded}

\section{RStudio and GitHub integration (6 /
7)}\label{rstudio-and-github-integration-6-7}

\section{RStudio and GitHub integration (7 /
7)}\label{rstudio-and-github-integration-7-7}

\section{Installing from GitHub}\label{installing-from-github}

\begin{Shaded}
\begin{Highlighting}[]
\NormalTok{devtools::}\KeywordTok{install_github}\NormalTok{(}\StringTok{"harvard-P01/pkgtemplate"}\NormalTok{)}
\end{Highlighting}
\end{Shaded}

\begin{Shaded}
\begin{Highlighting}[]
\NormalTok{devtools::}\KeywordTok{install_github}\NormalTok{(}\StringTok{"harvard-P01/pkgtemplate"}\NormalTok{,}
                         \DataTypeTok{build_vignettes =} \OtherTok{TRUE}\NormalTok{)}
\end{Highlighting}
\end{Shaded}

\section{\texorpdfstring{\texttt{.gitignore} (RStudio
default)}{.gitignore (RStudio default)}}\label{gitignore-rstudio-default}

\begin{Shaded}
\begin{Highlighting}[]
\NormalTok{.Rproj.user}
\NormalTok{.Rhistory}
\NormalTok{.RData}
\end{Highlighting}
\end{Shaded}

\section{\texorpdfstring{\texttt{.gitignore} (GitHub
default)}{.gitignore (GitHub default)}}\label{gitignore-github-default}

\begin{Shaded}
\begin{Highlighting}[]
\CommentTok{# History files}
\NormalTok{.Rhistory}
\NormalTok{.Rapp.history}

\CommentTok{# Example code in package build process}
\NormalTok{*-Ex.R}

\CommentTok{# RStudio files}
\NormalTok{.Rproj.user/}

\CommentTok{# produced vignettes}
\NormalTok{vignettes/}\ErrorTok{*}\NormalTok{.html}
\NormalTok{vignettes/}\ErrorTok{*}\NormalTok{.pdf}
\end{Highlighting}
\end{Shaded}

\section{RStudio projects}\label{rstudio-projects}

\begin{itemize}
\item
  \texttt{.Rproj} file extension, in our example
  \texttt{pkgtemplate.Rproj}
\item
  A project has its own:

  \begin{itemize}
  \tightlist
  \item
    R session
  \item
    .Rprofile (\emph{e.g.}, to customize startup environment)
  \item
    .Rhistory
  \end{itemize}
\item
  Default working directory is project directory
\item
  Keeps track of project-specific recent files
\end{itemize}

\section{Project options}\label{project-options}

\begin{Shaded}
\begin{Highlighting}[]
\NormalTok{Version:}\StringTok{ }\FloatTok{1.0}

\NormalTok{RestoreWorkspace:}\StringTok{ }\NormalTok{Default}
\NormalTok{SaveWorkspace:}\StringTok{ }\NormalTok{Default}
\NormalTok{AlwaysSaveHistory:}\StringTok{ }\NormalTok{Default}

\NormalTok{EnableCodeIndexing:}\StringTok{ }\NormalTok{Yes}
\NormalTok{UseSpacesForTab:}\StringTok{ }\NormalTok{Yes}
\NormalTok{NumSpacesForTab:}\StringTok{ }\DecValTok{2}
\NormalTok{Encoding:}\StringTok{ }\NormalTok{UTF}\DecValTok{-8}

\NormalTok{RnwWeave:}\StringTok{ }\NormalTok{knitr}
\NormalTok{LaTeX:}\StringTok{ }\NormalTok{pdfLaTeX}

\NormalTok{AutoAppendNewline:}\StringTok{ }\NormalTok{Yes}
\NormalTok{StripTrailingWhitespace:}\StringTok{ }\NormalTok{Yes}

\NormalTok{BuildType:}\StringTok{ }\NormalTok{Package}
\NormalTok{PackageUseDevtools:}\StringTok{ }\NormalTok{Yes}
\NormalTok{PackageInstallArgs:}\StringTok{ }\NormalTok{--no-multiarch --with-keep.source}
\end{Highlighting}
\end{Shaded}

\section{Package documentation}\label{package-documentation}

\begin{itemize}
\item
  Functions and methods
\item
  Vignettes

  \begin{itemize}
  \tightlist
  \item
    PDF
  \item
    \texttt{knitr} (or \texttt{Sweave})
  \end{itemize}
\end{itemize}

\section{Process example}\label{process-example}

Creating R Packages: A Tutorial (Friedrich Leisch, 2009 )

\begin{itemize}
\tightlist
\item
  \url{https://cran.r-project.org/doc/contrib/Leisch-CreatingPackages.pdf}
\end{itemize}

\section{\texorpdfstring{Adding \texttt{linreg.R} in \texttt{R/}
directory}{Adding linreg.R in R/ directory}}\label{adding-linreg.r-in-r-directory}

\begin{Shaded}
\begin{Highlighting}[]
\NormalTok{linmodEst <-}\StringTok{ }\NormalTok{function(x, y) \{}
  \NormalTok{## compute QR-decomposition of x}
  \NormalTok{qx <-}\StringTok{ }\KeywordTok{qr}\NormalTok{(x)}
  \NormalTok{## compute (x’x)^(-1) x’y}
  \NormalTok{coef <-}\StringTok{ }\KeywordTok{solve.qr}\NormalTok{(qx, y)}
  \NormalTok{## degrees of freedom and standard deviation of residuals}
  \NormalTok{df <-}\StringTok{ }\KeywordTok{nrow}\NormalTok{(x) -}\StringTok{ }\KeywordTok{ncol}\NormalTok{(x)}
  \NormalTok{sigma2 <-}\StringTok{ }\KeywordTok{sum}\NormalTok{((y -}\StringTok{ }\NormalTok{x %*%}\StringTok{ }\NormalTok{coef) ^}\StringTok{ }\DecValTok{2}\NormalTok{) /}\StringTok{ }\NormalTok{df}
  \NormalTok{## compute sigma^2 * (x’x)^-1}
  \NormalTok{vcov <-}\StringTok{ }\NormalTok{sigma2 *}\StringTok{ }\KeywordTok{chol2inv}\NormalTok{(qx$qr)}
  \KeywordTok{colnames}\NormalTok{(vcov) <-}\StringTok{ }\KeywordTok{rownames}\NormalTok{(vcov) <-}\StringTok{ }\KeywordTok{colnames}\NormalTok{(x)}
  \KeywordTok{list}\NormalTok{(}
    \DataTypeTok{coefficients =} \NormalTok{coef,}
    \DataTypeTok{vcov =} \NormalTok{vcov,}
    \DataTypeTok{sigma =} \KeywordTok{sqrt}\NormalTok{(sigma2),}
    \DataTypeTok{df =} \NormalTok{df}
  \NormalTok{)}
\NormalTok{\}}
\end{Highlighting}
\end{Shaded}

\section{Running our function}\label{running-our-function}

\begin{Shaded}
\begin{Highlighting}[]
\KeywordTok{data}\NormalTok{(cats, }\DataTypeTok{package =} \StringTok{"MASS"}\NormalTok{)}
\KeywordTok{linmodEst}\NormalTok{(}\KeywordTok{cbind}\NormalTok{(}\DecValTok{1}\NormalTok{, cats$Bwt), cats$Hwt)}
\end{Highlighting}
\end{Shaded}

\begin{verbatim}
## $coefficients
## [1] -0.3566624  4.0340627
## 
## $vcov
##            [,1]        [,2]
## [1,]  0.4792475 -0.17058197
## [2,] -0.1705820  0.06263081
## 
## $sigma
## [1] 1.452373
## 
## $df
## [1] 142
\end{verbatim}

\section{\texorpdfstring{And compare with \texttt{lm} (1 /
2)}{And compare with lm (1 / 2)}}\label{and-compare-with-lm-1-2}

\begin{Shaded}
\begin{Highlighting}[]
\NormalTok{lm1 <-}\StringTok{ }\KeywordTok{lm}\NormalTok{(Hwt ~}\StringTok{ }\NormalTok{Bwt, }\DataTypeTok{data=}\NormalTok{cats)}
\NormalTok{lm1}
\end{Highlighting}
\end{Shaded}

\begin{verbatim}
## 
## Call:
## lm(formula = Hwt ~ Bwt, data = cats)
## 
## Coefficients:
## (Intercept)          Bwt  
##     -0.3567       4.0341
\end{verbatim}

\begin{Shaded}
\begin{Highlighting}[]
\KeywordTok{coef}\NormalTok{(lm1)}
\end{Highlighting}
\end{Shaded}

\begin{verbatim}
## (Intercept)         Bwt 
##  -0.3566624   4.0340627
\end{verbatim}

\section{\texorpdfstring{And compare with \texttt{lm} (2 /
2)}{And compare with lm (2 / 2)}}\label{and-compare-with-lm-2-2}

\begin{Shaded}
\begin{Highlighting}[]
\KeywordTok{vcov}\NormalTok{(lm1)}
\end{Highlighting}
\end{Shaded}

\begin{verbatim}
##             (Intercept)         Bwt
## (Intercept)   0.4792475 -0.17058197
## Bwt          -0.1705820  0.06263081
\end{verbatim}

\begin{Shaded}
\begin{Highlighting}[]
\KeywordTok{summary}\NormalTok{(lm1)$sigma}
\end{Highlighting}
\end{Shaded}

\begin{verbatim}
## [1] 1.452373
\end{verbatim}

\section{Adding ROxygen2
documentation}\label{adding-roxygen2-documentation}

\begin{Shaded}
\begin{Highlighting}[]
\CommentTok{#' Linear regression}
\CommentTok{#'}
\CommentTok{#' Runs an OLS regression not unlike \textbackslash{}code\{\textbackslash{}link\{lm\}\}}
\CommentTok{#'}
\CommentTok{#' @param y response vector (1 x n)}
\CommentTok{#' @param X covariate matrix (p x n) with no intercept}
\CommentTok{#'}
\CommentTok{#' @return A list with 4 elements: coefficients, vcov, sigma, df}
\CommentTok{#'}
\CommentTok{#' @examples}
\CommentTok{#' data(mtcars)}
\CommentTok{#' X <- as.matrix(mtcars[, c("cyl", "disp", "hp")])}
\CommentTok{#' y <- mtcars[, "mpg"]}
\CommentTok{#' linreg(y, X)}
\CommentTok{#'}
\CommentTok{#' @export}
\CommentTok{#'}
\NormalTok{linmodEst <-}\StringTok{ }\NormalTok{function(x, y) \{}
  \NormalTok{## compute QR-decomposition of x}
  \NormalTok{qx <-}\StringTok{ }\KeywordTok{qr}\NormalTok{(x)}
  \NormalTok{## compute (x’x)^(-1) x’y}
  \NormalTok{coef <-}\StringTok{ }\KeywordTok{solve.qr}\NormalTok{(qx, y)}
  \NormalTok{## degrees of freedom and standard deviation of residuals}
  \NormalTok{df <-}\StringTok{ }\KeywordTok{nrow}\NormalTok{(x) -}\StringTok{ }\KeywordTok{ncol}\NormalTok{(x)}
  \NormalTok{sigma2 <-}\StringTok{ }\KeywordTok{sum}\NormalTok{((y -}\StringTok{ }\NormalTok{x %*%}\StringTok{ }\NormalTok{coef) ^}\StringTok{ }\DecValTok{2}\NormalTok{) /}\StringTok{ }\NormalTok{df}
  \NormalTok{## compute sigma^2 * (x’x)^-1}
  \NormalTok{vcov <-}\StringTok{ }\NormalTok{sigma2 *}\StringTok{ }\KeywordTok{chol2inv}\NormalTok{(qx$qr)}
  \KeywordTok{colnames}\NormalTok{(vcov) <-}\StringTok{ }\KeywordTok{rownames}\NormalTok{(vcov) <-}\StringTok{ }\KeywordTok{colnames}\NormalTok{(x)}
  \KeywordTok{list}\NormalTok{(}
    \DataTypeTok{coefficients =} \NormalTok{coef,}
    \DataTypeTok{vcov =} \NormalTok{vcov,}
    \DataTypeTok{sigma =} \KeywordTok{sqrt}\NormalTok{(sigma2),}
    \DataTypeTok{df =} \NormalTok{df}
  \NormalTok{)}
\NormalTok{\}}
\end{Highlighting}
\end{Shaded}

\section{Configure Build Tools}\label{configure-build-tools}

\section{\texorpdfstring{\texttt{man/linmodEst.Rd}}{man/linmodEst.Rd}}\label{manlinmodest.rd}

\begin{Shaded}
\begin{Highlighting}[]
\NormalTok{% Generated by }\KeywordTok{roxygen2} \NormalTok{(}\FloatTok{4.1.1}\NormalTok{):}\StringTok{ }\NormalTok{do not edit by hand}
\NormalTok{% Please edit documentation in R/linmodEst.R}
\NormalTok{\textbackslash{}name\{linmodEst\}}
\NormalTok{\textbackslash{}alias\{linmodEst\}}
\NormalTok{\textbackslash{}title\{Linear regression\}}
\NormalTok{\textbackslash{}usage\{}
\KeywordTok{linmodEst}\NormalTok{(x, y)}
\NormalTok{\}}
\NormalTok{\textbackslash{}arguments\{}
\NormalTok{\textbackslash{}item\{y\}\{response }\KeywordTok{vector} \NormalTok{(}\DecValTok{1} \NormalTok{x n)\}}

\NormalTok{\textbackslash{}item\{X\}\{covariate }\KeywordTok{matrix} \NormalTok{(p x n) with no intercept\}}
\NormalTok{\}}
\NormalTok{\textbackslash{}value\{}
\NormalTok{A list with }\DecValTok{4} \NormalTok{elements:}\StringTok{ }\NormalTok{coefficients, vcov, sigma, df}
\NormalTok{\}}
\NormalTok{\textbackslash{}description\{}
\NormalTok{Runs an OLS regression not unlike \textbackslash{}code\{\textbackslash{}link\{lm\}\}}
\NormalTok{\}}
\NormalTok{\textbackslash{}examples\{}
\KeywordTok{data}\NormalTok{(mtcars)}
\NormalTok{X <-}\StringTok{ }\KeywordTok{as.matrix}\NormalTok{(mtcars[, }\KeywordTok{c}\NormalTok{(}\StringTok{"cyl"}\NormalTok{, }\StringTok{"disp"}\NormalTok{, }\StringTok{"hp"}\NormalTok{)])}
\NormalTok{y <-}\StringTok{ }\NormalTok{mtcars[, }\StringTok{"mpg"}\NormalTok{]}
\KeywordTok{linmodEst}\NormalTok{(y, X)}
\NormalTok{\}}
\end{Highlighting}
\end{Shaded}

\section{Formatted output}\label{formatted-output}

\section{\texorpdfstring{\texttt{DESCRIPTION}}{DESCRIPTION}}\label{description}

\begin{Shaded}
\begin{Highlighting}[]
\NormalTok{Package:}\StringTok{ }\NormalTok{pkgtemplate}
\NormalTok{Type:}\StringTok{ }\NormalTok{Package}
\NormalTok{Title:}\StringTok{ }\NormalTok{What the Package }\KeywordTok{Does} \NormalTok{(Title Case)}
\NormalTok{Version:}\StringTok{ }\FloatTok{0.1}
\NormalTok{Date:}\StringTok{ }\DecValTok{2015-10-24}
\NormalTok{Author:}\StringTok{ }\NormalTok{Who wrote it}
\NormalTok{Maintainer:}\StringTok{ }\NormalTok{Who to complain to <yourfault@somewhere.net>}
\NormalTok{Description:}\StringTok{ }\NormalTok{More about what it }\KeywordTok{does} \NormalTok{(maybe more than one line)}
\NormalTok{License:}\StringTok{ }\NormalTok{What license is it under?}
\NormalTok{LazyData:}\StringTok{ }\OtherTok{TRUE}
\end{Highlighting}
\end{Shaded}

\section{\texorpdfstring{\texttt{NAMESPACE}}{NAMESPACE}}\label{namespace}

\texttt{export}'s automatically generated when parsing ROxygens2
snippets

\begin{Shaded}
\begin{Highlighting}[]
\KeywordTok{export}\NormalTok{(linmodEst)}
\end{Highlighting}
\end{Shaded}

\section{S3 basics}\label{s3-basics}

\begin{Shaded}
\begin{Highlighting}[]
\NormalTok{hello <-}\StringTok{ }\NormalTok{function() \{}
 \NormalTok{s <-}\StringTok{ "Hello World!"}
 \KeywordTok{class}\NormalTok{(s) <-}\StringTok{ "hi"}
 \KeywordTok{return}\NormalTok{(s)}
\NormalTok{\}}

\KeywordTok{hello}\NormalTok{()}
\end{Highlighting}
\end{Shaded}

\begin{verbatim}
## [1] "Hello World!"
## attr(,"class")
## [1] "hi"
\end{verbatim}

\section{S3 basics}\label{s3-basics-1}

\begin{Shaded}
\begin{Highlighting}[]
\NormalTok{print.hi <-}\StringTok{ }\NormalTok{function(...) \{}
  \KeywordTok{print}\NormalTok{(}\StringTok{"Surprise!"}\NormalTok{)}
\NormalTok{\}}

\KeywordTok{hello}\NormalTok{()}
\end{Highlighting}
\end{Shaded}

\begin{verbatim}
## [1] "Surprise!"
\end{verbatim}

\section{S3 and S4 generics}\label{s3-and-s4-generics}

\begin{Shaded}
\begin{Highlighting}[]
\NormalTok{linmod <-}\StringTok{ }\NormalTok{function(x, ...)}
  \KeywordTok{UseMethod}\NormalTok{(}\StringTok{"linmod"}\NormalTok{)}
\end{Highlighting}
\end{Shaded}

\begin{Shaded}
\begin{Highlighting}[]
\NormalTok{linmod.default <-}\StringTok{ }\NormalTok{function(x, y, ...) \{}
  \NormalTok{x <-}\StringTok{ }\KeywordTok{as.matrix}\NormalTok{(x)}
  \NormalTok{y <-}\StringTok{ }\KeywordTok{as.numeric}\NormalTok{(y)}
  \NormalTok{est <-}\StringTok{ }\KeywordTok{linmodEst}\NormalTok{(x, y)}
  \NormalTok{est$fitted.values <-}\StringTok{ }\KeywordTok{as.vector}\NormalTok{(x %*%}\StringTok{ }\NormalTok{est$coefficients)}
  \NormalTok{est$residuals <-}\StringTok{ }\NormalTok{y -}\StringTok{ }\NormalTok{est$fitted.values}
  \NormalTok{est$call <-}\StringTok{ }\KeywordTok{match.call}\NormalTok{()}
  \KeywordTok{class}\NormalTok{(est) <-}\StringTok{ "linmod"}
  \KeywordTok{return}\NormalTok{(est)}
\NormalTok{\}}
\end{Highlighting}
\end{Shaded}

\section{\texorpdfstring{\texttt{print}}{print}}\label{print}

\begin{Shaded}
\begin{Highlighting}[]
\NormalTok{print.linmod <-}\StringTok{ }\NormalTok{function(x, ...) \{}
  \KeywordTok{cat}\NormalTok{(}\StringTok{"Call:}\CharTok{\textbackslash{}n}\StringTok{"}\NormalTok{)}
  \KeywordTok{print}\NormalTok{(x$call)}
  \KeywordTok{cat}\NormalTok{(}\StringTok{"}\CharTok{\textbackslash{}n}\StringTok{Coefficients:}\CharTok{\textbackslash{}n}\StringTok{"}\NormalTok{)}
  \KeywordTok{print}\NormalTok{(x$coefficients)}
\NormalTok{\}}
\end{Highlighting}
\end{Shaded}

\section{\texorpdfstring{\texttt{print}}{print}}\label{print-1}

\begin{Shaded}
\begin{Highlighting}[]
\NormalTok{x <-}\StringTok{ }\KeywordTok{cbind}\NormalTok{(}\DataTypeTok{Const =} \DecValTok{1}\NormalTok{, }\DataTypeTok{Bwt =} \NormalTok{cats$Bwt)}
\NormalTok{y <-}\StringTok{ }\NormalTok{cats$Hw}
\NormalTok{mod1 <-}\StringTok{ }\KeywordTok{linmod}\NormalTok{(x, y)}
\NormalTok{mod1}
\end{Highlighting}
\end{Shaded}

\begin{verbatim}
## Call:
## linmod.default(x = x, y = y)
## 
## Coefficients:
##      Const        Bwt 
## -0.3566624  4.0340627
\end{verbatim}

\section{Other methods}\label{other-methods}

\begin{itemize}
\item
  \texttt{summary.linmod}
\item
  \texttt{print.summary.linmod}
\item
  \texttt{predict.linmod}
\item
  \texttt{plot.linmod}
\item
  \texttt{coef.linmod}, \texttt{vcov.linmod}, \ldots{}
\end{itemize}

\section{Formulas and model frames}\label{formulas-and-model-frames}

\begin{Shaded}
\begin{Highlighting}[]
\NormalTok{linmod.formula <-}\StringTok{ }\NormalTok{function(formula, }\DataTypeTok{data =} \KeywordTok{list}\NormalTok{(), ...) \{}
  \NormalTok{mf <-}\StringTok{ }\KeywordTok{model.frame}\NormalTok{(}\DataTypeTok{formula =} \NormalTok{formula, }\DataTypeTok{data =} \NormalTok{data)}
  \NormalTok{x <-}\StringTok{ }\KeywordTok{model.matrix}\NormalTok{(}\KeywordTok{attr}\NormalTok{(mf, }\StringTok{"terms"}\NormalTok{), }\DataTypeTok{data =} \NormalTok{mf)}
  \NormalTok{y <-}\StringTok{ }\KeywordTok{model.response}\NormalTok{(mf)}
  \NormalTok{est <-}\StringTok{ }\KeywordTok{linmod.default}\NormalTok{(x, y, ...)}
  \NormalTok{est$call <-}\StringTok{ }\KeywordTok{match.call}\NormalTok{()}
  \NormalTok{est$formula <-}\StringTok{ }\NormalTok{formula}
  \KeywordTok{return}\NormalTok{(est)}
\NormalTok{\}}
\end{Highlighting}
\end{Shaded}

\section{\texorpdfstring{Unit tests and
\texttt{testthat}}{Unit tests and testthat}}\label{unit-tests-and-testthat}

\url{http://r-pkgs.had.co.nz/tests.html}

In package directory:

\begin{Shaded}
\begin{Highlighting}[]
\NormalTok{devtools::}\KeywordTok{use_testthat}\NormalTok{()}
\end{Highlighting}
\end{Shaded}

pre-populates \texttt{test/testthat/}

Test files should start with \texttt{test} to be processed.

\section{\texorpdfstring{\texttt{test\_coef.R}}{test\_coef.R}}\label{test_coef.r}

\begin{Shaded}
\begin{Highlighting}[]
\KeywordTok{data}\NormalTok{(cats, }\DataTypeTok{package =} \StringTok{"MASS"}\NormalTok{)}
\NormalTok{l1 <-}\StringTok{ }\KeywordTok{linmod}\NormalTok{(Hwt ~}\StringTok{ }\NormalTok{Bwt *}\StringTok{ }\NormalTok{Sex, }\DataTypeTok{data =} \NormalTok{cats)}
\NormalTok{l2 <-}\StringTok{ }\KeywordTok{lm}\NormalTok{(Hwt ~}\StringTok{ }\NormalTok{Bwt *}\StringTok{ }\NormalTok{Sex, }\DataTypeTok{data =} \NormalTok{cats)}

\KeywordTok{test_that}\NormalTok{(}\StringTok{"same estimated coefficients as lm function"}\NormalTok{, \{}
  \KeywordTok{expect_equal}\NormalTok{(l1$coefficients, l2$coefficients)}
\NormalTok{\})}
\end{Highlighting}
\end{Shaded}

\begin{Shaded}
\begin{Highlighting}[]
\NormalTok{==}\ErrorTok{>}\StringTok{ }\NormalTok{devtools::}\KeywordTok{test}\NormalTok{()}

\NormalTok{Loading pkgtemplate}
\NormalTok{Loading required package:}\StringTok{ }\NormalTok{testthat}
\NormalTok{Testing pkgtemplate}
\NormalTok{.}
\NormalTok{Woot!}\StringTok{ }
\end{Highlighting}
\end{Shaded}

\section{Vignettes}\label{vignettes}

\url{http://r-pkgs.had.co.nz/vignettes.html}

\begin{Shaded}
\begin{Highlighting}[]
\NormalTok{devtools::}\KeywordTok{use_vignette}\NormalTok{(}\StringTok{"linmod"}\NormalTok{)}
\end{Highlighting}
\end{Shaded}

\url{https://github.com/harvard-P01/pkgtemplate/blob/master/vignettes/linmod.Rmd}

\chapter{Optimization}\label{optimization}

In this Chapter, we are going to see how to measure and improve code
performance

\section{Measuring performance}\label{measuring-performance}

\subsection{Profiling}\label{profiling}

\subsection{Benchmarking}\label{benchmarking}

\section{Improving performance}\label{improving-performance}

\subsection{Introduction to C/C++}\label{introduction-to-cc}

\subsection{Rcpp}\label{rcpp}

\chapter{Databases}\label{databases}

\section{Overview}\label{overview}

\section{SQL}\label{sql}

\section{noSQL}\label{nosql}

\section{R interfaces}\label{r-interfaces}

\chapter{Big data}\label{bigdata}

In this Chapter, we are going to review different approaches to handle
and perform analyses on \emph{data that does not fit in memory}.

\section{Reading big data (that fits in
memory)}\label{reading-big-data-that-fits-in-memory}

\subsection{R package comparison}\label{r-package-comparison}

\subsection{Python}\label{python}

\section{Sampling (can be read, not analyzed
easily)}\label{sampling-can-be-read-not-analyzed-easily}

\section{Pure R solutions}\label{pure-r-solutions}

\section{JVM solutions}\label{jvm-solutions}

\subsection{h20}\label{h20}

\subsection{Spark}\label{spark}

\chapter{Visualization}\label{visualization}

We have finished a nice book.

\bibliography{packages.bib,book.bib}


\end{document}
